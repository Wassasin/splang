\documentclass[14pt]{amsart}

\usepackage[british]{babel}

\usepackage{graphicx}
\usepackage{amssymb}
\usepackage{color}
\usepackage{listings}
\usepackage{xspace}
\usepackage{url}

\newcommand{\splang}{\textsc{Splang}\xspace}

\title{\splang}
\author{Wouter Geraedts (0814857) \and Joshua Moerman (3009048)}
\date{}

\begin{document}
\maketitle

\tableofcontents

\section{Implementation}
For our SPL-compiler, \splang, we choose \emph{Haskell} as our implementation language. First of all because this is a \emph{functional programming language}, which is really helpful when defining datastructures. Another nice things is the \emph{\texttt{do}-syntax} of haskell, which makes it very easy (at least to some extend) to propagate for example error messages. Of course this part would also be very easy in a imperitave language.

One feature we miss out in Haskell is for example \emph{generics} as in \emph{Clean} (however the \texttt{deriving Functor} possibility of Haskell \emph{is} nice). Of course there are also some minor irritations because of some differences between Clean and Haskell.

We depend on two libraries (other than the library which comes with Haskell), namely \emph{ansi-terminal} and \emph{edit-distance}. We use those for (resp.) coloured output in the terminal (colour scheme is based on the {\sc Clang} compiler) and suggestions when the programmer typed a undeclared identifier.

\section{Parser}
We made our own parsing combinator library. It has infinite lookahead, but when performance matters there are also 1-lookahead combinators. We choose to make our own one, because then we would have full control of error messages and sourecode information. It is implemented in a monadic way.

\subsection{Grammar}
In order to parse the language we had to change the grammar a bit. First of all we added priorities to the different operators, so that the multiplication binds stronger than addition.

Secondly, we generalized the grammar a bit, not only function calls can be an assignment, but also expressions.

We also noted that the given grammar was ambiguous, because of the \emph{dangling else}. We didn't have to change this, nor choose an convention. Our parser has infinite lookahead, so we parse all possibilities and conclude that it is ambiguous, and throw an error.

At last there is the problem of left-recursion in the given grammar. Our parser is left-descent, so we had to do left-factoring. We also had to think about associativity of non-associative operators.

The used grammar is attached in the appendix.

\section{Scoping}
We distinguish three different scopes (from big to small): \emph{global}, \emph{function argument} and \emph{local}. If two identifiers are being declared in the same scope, an error will be thrown, but compilation will continue (to possibly catch more errors). If a identifier is being declared, but it was already declared in a bigger scope, we allow this, but we will give the user a warning (because it is probably not what you want, and might introduce subtle bugs). This is called shadowing, one can still initialize this shadowing identifier with the previous one, as shown here:

\begin{lstlisting}
Int x = 5; // Global
Void foo(){
    Int x = x; // ``Warning: x shadows global x''
               // x will be still initialized with the global x,
               // ie.: x == 5
}
\end{lstlisting}

Globals can be defined in any order you want. This for example allows mutual recursion. This also holds for variable declarations, so one can declare a variable using variables which are defined later. But the compiler does not make a dependency graph, so using a variable which is not defined yet, will result in using uninitialized memory.

As far as scoping is concerned variables can be used as a function, and functions can be passed to other functions. However the type system does not accept this, because the annotated type in the source can never be a function type.

\subsection{Type variable scoping}
Type variables have similar scoping rules. If two typevariables are used in the same scope, they will be exactly the same type. In the following example \texttt{x} and \texttt{y} are of the same type, and therefore the second \texttt{print} will fail to typecheck. (We will later see another reason why this program is not allowed by the compiler).

\begin{lstlisting}
[t] x = [];
[t] y = []; // Same type as x

Void main(){
	print(1:x);    // => x is of type [Int]
	print(True:y); // => y is of type [Bool] => error
}
\end{lstlisting}

There is however an exception to this. We assume that if a function uses a type variable, then it is a polymorphic function. This means that type variables in function arguments are always new, in fact, this is the only way to create a new type variable with the same name. For example:

\begin{lstlisting}
t x = /* ... */

// The following 't' is independent of the above
[t] reverse([t] list){
	/* ... */
}
\end{lstlisting}

Note that type variables and identifiers for variables and functions live in two seperate namespaces, so there is never a name clash. (For example we allow: \texttt{x x(x x)\{return x;\}}, which is the identity function.)

\section{Typing}

Our compiler mainly does typechecking, but some type inference is also needed. Type inference is hard to define in a imperitave language, as the variables one is infering, might be used differently later on (and mutual recursion is hard). This led to a system which does both type inference and type checking.

We first determine the annotated types of every global variable and function. Then in all expressions and function bodies those annotations are used to infer the actual type at every place in the expression or function body, and with this information the annotated type is checked. We also make sure that \texttt{return} statements really return what they should, and if there is no \texttt{return} statement, the function should return \texttt{Void}.

We only allow \texttt{Void} to occur as a return type of a function. So a list or tuples of \texttt{Void} will give an error. One also cannot supply \texttt{Void} as argument, even when the function is polymorphic. For example \texttt{print(print(5))} will not compile for this reason.

We do not allow the programmer to write this:

\begin{lstlisting}
t x = 5;
\end{lstlisting}

Because it is weird to annotate x with a more general type than it really is. This is a explicit decision we made, technically the compiler is able to infer that \texttt{t == Int}. Allowing this code for variables might seem reasonable, but allowing this also for functions is wrong, as can be seen in the following example:

\begin{lstlisting}
Bool x = f();

// This should not be allowed
t f(){ return 5; }
\end{lstlisting}

Type checking the assignment of the global variable \texttt{y} is ok, since \texttt{f} returns everything you want. So we should not allow the type annotation given to the function \texttt{f}. For consistency we also enforce this for variables. We might at some point in the future allow this flexibility for variables.

\subsection{Formal rules}
at funcalls, use type annotation
at fundecl, initialize to free type variables
when fundecl is done inferring, check result with type annotation (while checking, must exists a bijection between ftv's)

\newcommand{\T}{\mathcal{T}}
\newcommand{\U}{\mathcal{U}}
\newcommand{\s}{\ast}
\newcommand{\sn}[1]{{\s_{#1}}}

% $\T(\Gamma, x, \sigma) = \U(\tau[\vec{\alpha} := \vec{\beta}], \sigma) \circ \ast$
	
$\begin{array}{rcl}
	\T_{\mbox{d}}(\Gamma, \mbox{Program}~\vec{d}) & = & \s_n \\
	\sn{n} & = & \T_{\mbox{d}}(\Gamma^{\sn{n-1}}, d_n) \circ \sn{n-1} \\
	& \vdots & \\
	\sn{2} & = & \T_{\mbox{d}}(\Gamma^{\sn{1}}, d_2) \circ \sn{1} \\
	\sn{1} & = & \T_{\mbox{d}}(\Gamma, d_1) \\
\end{array}$

$\begin{array}{rcl}
	\T_{\mbox{d}}(\Gamma, \mbox{VarDecl}~t~x~e) & = & \T(\Gamma, e, \tau) \\
	\forall \emptyset . \tau	& = & \Gamma(x) \\
\end{array}$

$\begin{array}{rcl}
	\T_{\mbox{d}}(\Gamma, \mbox{FunDecl}~t~x~\vec{\mbox{arg}}~\vec{\mbox{stmt}}) & = & \U(\theta^\sn{1}, \upsilon^\sn{1}) \circ \sn{1} \\
	\upsilon	& = & \vec{\alpha}^\sn{1} \rightarrow \beta^\sn{1} \\
	\sn{1}		& = & \sn{(2,n)} \\
	\sn{(2,n)}	& = & \T(\Gamma^\sn{(2,n-1)}, \mbox{stmt}_n, \beta^\sn{(2,n-1)}) \circ \sn{(2,n-1)} \\
	& \vdots & \\
	\sn{(2,2)}	& = & \T(\Gamma^\sn{(2,1)}, \mbox{stmt}_2, \beta^\sn{(2,1)}) \circ \sn{(2,2)} \\
	\sn{(2,1)}	& = & \T(\Gamma^\sn{3}, \mbox{stmt}_1, \beta^\sn{3}) \circ \sn{3} \\
	
	
\end{array}$



\section{Tests}
All tests are performed with the \texttt{--show-input --show-stages} flags, and all colors are stripped. The output is a pretty printed version of the program augmented with both the scoping results (after every identifier), and typing (before every declaration). For every \texttt{.spl} file, there is a \texttt{.spl.txt} file with gray output. The coloured output is also attached in a different pdf file.

\subsection{Parsing}
\begin{itemize}
	\item[fail\_ambi] This program can be read ambiguously. Note that the compiler detects this, and prints out all possible interpretations. However this error is not fatal, and the compilation continues, and catches a typing error: the \texttt{if}-construct expects a \texttt{Bool}, but an \texttt{Int} was given.
	\item[pass\_parser] This program shows that the compiler correctly handles associativity and priorities of infix operators.
\end{itemize}
\subsection{Scoping}
\begin{itemize}
	\item[warn\_shadowing] This shows the warnings one get when redeclaring the same identifier in a more specific scope. Note that the functions \texttt{x} and \texttt{y} are both identity functions.
	\item[fail\_identifier\_errors] If one redeclares identifiers in the same scope, an non-fatal error will be given. Note that compilation continues and more errors are found. If an identifiers is undeclared, a suggestion is given by the compiler.
\end{itemize}
\subsection{Typing}
\begin{itemize}
	\item[fail\_arguments] Shows the error messages you get when you don't suply the right amount of arguments. Also note that the arguments that \emph{are} supplied, are also type checked.
	\item[fail\_void\_no\_return] Shows that we do not accept a returning function without return statement. Specifically, we see that the compiler infers that \texttt{foo} returns \texttt{Void}, instead of \texttt{Int}. Furthermore we see that we cannot use values of type \texttt{Void}.
	\item[pass\_merge\_sort] Shows that the compiler is capable of handling polymorphism. And is nicely shows a real world example, and also shows mutual recursion.
	\item[pass\_reverse] Shows a basic reverse function. It is also used on a list of lists.
	\item[fail\_empty\_list] Shows our compilers output on the \emph{Empty list} example discussed above. Note that the type \texttt{t} is used in different ways (both \texttt{Int} and \texttt{Bool}).
	\item[pass\_polymorphism] Shows polymorphism and also scoping of type variables.
\end{itemize}

\section{Reflection}
In the first phase Wouter made the lexer, and Joshua made a first implementation of the parser in \emph{Parsec}, however we decided to write our own parsing library. This was mainly done by Wouter. In the second phase Joshua mainly worked on the scoping and type annotations, Wouter mainly on the type inference. All design choices were discussed together to ensure a working whole. And also in some cases one found a bug in the others code. So eventually all code understood by both of us.

For a very precise description of who did what, one can have a look at our repository:

\url{https://github.com/Wassasin/splang}

\newpage
\appendix
\section{Grammar}

\newcommand{\tok}[1]{`\texttt{#1}'}
\newcommand{\I}{\hspace{0.1cm}$\mid$\hspace{0.2cm}}

\begin{tabular}[t]{p{2.5cm} c p{10cm}}
Prog		& := & Decl$^+$					\\
Decl		& := & VarDecl \I FunDecl			\\
VarDecl		& := & Type id \tok{=} Exp \tok{;}		\\
FunDecl		& := & RetType id \tok{(} [ FArgs ] \tok{)} \tok{\{} VarDecl* Stmt$^+$ \tok{\}} \\
RetType		& := & Type \I \tok{Void}			\\
Type		& := & \tok{Int} \I \tok{Bool} \I id		\\
		& \I & \tok{(} Type \tok{,} Type \tok{)}	\\
		& \I & \tok{[} Type \tok{]}			\\
FArgs		& := & Type id \I Type id \tok{,} Fargs		\\
&&\\
Stmt		& := & \tok{\{} Stmt* \tok{\}}			\\
		& \I & \tok{if} \tok{(} Exp \tok{)} Stmt	\\
		& \I & \tok{if} \tok{(} Exp \tok{)} Stmt \tok{else} Stmt \\
		& \I & \tok{while} \tok{(} Exp \tok{)} Stmt 	\\
		& \I & id \tok{=} Exp \tok{;}			\\
		& \I & Exp \tok{;}				\\
		& \I & \tok{return} Exp \tok{;}			\\
&&\\
Exp		& := & Term0					\\
Term0		& := & Term1 \I Term1 Op2Bool Term0		\\
Term1		& := & Term2 \I Term2 Op2Equal Term1		\\
Term2		& := & Term3 \I OpNot Term2			\\
Term3		& := & Term4 \I Term4 OpCons Term3		\\
Term4		& := & Term5 \I Term5 Term4b			\\
Term4b		& := & Op2Add Term5 Term4b \I $\epsilon$	\\
Term5		& := & Term6 \I Term6 Term5b			\\
Term5b		& := & Op2Mult Term6 Term5b \I $\epsilon$	\\
Term6		& := & OpNegative Term6 \I Term7		\\
Term7		& := & int					\\
		& \I & \tok{(} Exp \tok{)}			\\
		& \I & \tok{(} Exp \tok{,} Exp \tok{)}		\\
		& \I & \tok{False}				\\
		& \I & \tok{True}				\\
		& \I & id					\\
		& \I & FunCall					\\
		& \I & \tok{[]}					\\
&&\\
FunCall		& := & id \tok{(} [ ActArgs ] \tok{)}		\\
ActArgs		& := & Exp \I Exp \tok{,} ActArgs		\\
&&\\
Op2Mult		& := & \tok{$\ast$} \I \tok{/} \I \tok{\%} 	\\
Op2Add		& := & \tok{+} \I \tok{-}			\\
Op2Cons		& := & \tok{:}					\\
Op2Equal	& := & \tok{==} \I \tok{<} \I \tok{>} \I \tok{<=} \I \tok{>=} \I \tok{!=} \\
Op2Bool		& := & \tok{\&\&} \I \tok{||}			\\
&&\\
OpNot		& := & \tok{!}					\\
OpNegative	& := & \tok{-}					\\
&&\\
int		& := & digit$^+$				\\
id		& := & alpha \I alpha id'			\\
id'		& := & id' \tok{\_} \I id' alphaNum
\end{tabular}


\end{document}
